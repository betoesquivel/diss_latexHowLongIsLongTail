\section{Related Work} %RW
\subsection{Entity linking overview} %ELO
% Describe what entity recognition is
% -What is entity recognition
% Describe current entity linking approaches
% -What is entity linking
% -Popular entity linkers available
% Dbpedia algorithm description
% -As an example, DBPedia algorithm description
%% How is this related: It is statistical, so it generates long-tail entities.
%% How I build on this: Proper description for the long-tail entity set and measurement? 

%%%%%%%%%%
% Describe what entity recognition is
% -What is entity recognition
% -Stanford NER and Open NLP description
Entity Linking is an Information Extraction (IE) process by which named entities in text are identified and linked to a knowledge base.

The first step in this task, identifying the named entities, belongs to a category of its own, Named Entity Recognition.
\cite{rw_elo_Nadeau2009} best describes the purpose of this task as the identification of entities with "rigid designators",
such as proper names and terms. 



%%%%%%%%%%
% Describe current entity linking approaches
% -What is entity linking
% -Popular entity linkers available

%%%%%%%%%%
% Dbpedia algorithm description
% -As an example, DBPedia algorithm description

%%%%%%%%%%
%% How is this related: It is statistical, so it generates long-tail entities.
%% How I build on this: Proper description for the long-tail entity set and measurement? 

\subsection{Infrastructure requirements for entity linking}
% Ways of using entity linkers at the moment. Use their REST service or deploy your own. 
% Memory requirements
%% How is this related: Difficult to conduct research without knowledge of infrastructure cluster management or a server that satisfies this requirements. 
%% How I build on this: Build a system that uses Cloud-based technologies for tagging data and querying the results. 
\subsection{NIL Entities versus Long-tail Entities}
% How are NIL Entities described in current papers.
% What work is been done with them at the moment.
%% Justification of differentiating between NIL Entities and Long-tail Entities
%% Long-tail entities are specific to their domain. They must be identified isolated within their domain.
%% Commercial relevance of Long-tail entities suggestion (show commercial applications that do Long-tail entity recognition)
%% Describing and measuring the set of Long-tail entities allows businesses who benefit from IE tasks to decide if they will benefit from performing more granular IE tasks, which can be expensive (supervised learning of Long-tail entities requires labeling of data) to identify Long-tail entities.
