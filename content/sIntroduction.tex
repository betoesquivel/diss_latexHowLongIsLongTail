\section{Introduction}
% Importance of information extraction
% Summarized overview of current approaches to IR
% Entity linking in IR
% State of the art entity linking limitations -> Long-tail Entities
% Objs
% 1. To establish a base on which further research can be done on Long-tail Entity linking
%% Suggesting a description for Long-tail entities
%% Describing a scalable no-management system for comparing text tagging applications
% 2. To present a practical description of Long-tail entities in the News domain
%% Present a method for measuring the size of the Long-tail entity set in the News domain
%% Carry out presented method on real data from the News domain (Signal Media's 1M dataset)
The growth of unstructured textual information available online has attracted the attention of the Natural Language Processing (NLP) research community.
Since the advent of the internet, plenty of research has been done on the area of Information Extraction (IE) from text,
and whenever an IE problem is established by the community, work usually starts with extracting entities from text.

The process of recognizing entities (Entity Recognition) and linking them (Entity Linking) to structured repositories of information in text
is a common preceding step before many Information Extraction tasks.
For instance, search is a good example of this, where over 70\% of searches or queries are entity-based or at least keyword-based \cite{webpatterns},
and around 64\% of queries tend to be unique.
In this case, grouping queries becomes a hard problem because they are so different or, better said, the set of queries is sparse.
Therefore, an improvement on the identification of entities in queries would mean an overall improvement on the search task,
since queries could be grouped by entity or entity type.

Also, this task has recently become highly relevant in the commercial environment, with many companies trying to make sense of their unstructured textual data
in order to gain some insight that may yield a competitive advantage;
information extraction and text analytics has even become a popular area for startup businesses to compete in \cite{techcrunch, ventureradar},
with even the big companies in the industry (Apple, Google, Microsoft, Yahoo, ...) all participating with their own products that leverage IE technologies and research.

However, evaluation techniques for Entity Linking tasks could be suffering from the same problem that search has suffered\cite{webpatterns}, where the quality
of the retrieved links is the main focus of the research, leading to task improvements only in the recognition of common popular entities, not focusing on the
often times harder problem of recognizing and linking the uncommon entities that rarely appear in text.

Many approaches to Entity Linking are probabilistic \cite{probabilistic, rw_elo_morsey2012dbpedia, rw_elo_Nadeau2009},
achieving a state-of-the art efficiency by approximating links using statistics of the corpus and characteristics of huge Knowledge Bases (KB) like DBPedia
\cite{rw_elo_morsey2012dbpedia} to solve the task.
However, given the everchanging nature of the existing set of entities, often the conclusions of these approaches, like \cite{probabilistic}, end up suggesting
the addition of information from other KBs or sources to improve the coverage of the entities they can recognise.
Therefore, the state of the art Entity Linking (EL) applications end up consuming a large amount of computing and memory resources, making it hard to perform new research on the area.
In addition to this, evaluating an Entity Linker on a real dataset big enough to provide significant results ends up been an overwhelming and slow task.

To solve these problems, the purpose of this paper was to establish a base to facilitate further research on Long-tail Entity Linking
in a way that is practical and accessible to new researchers in the area.
The main contributions of this paper are:
\begin{enumerate}
  \item Provided a description with real examples of Long-tail entity types in newswire data.
  \item Built and described a scalable no-management system for comparing text tagging applications, which also proved accessible for starting researchers in the area and educators by leveraging cloud technologies.
  \item Presented a practical example of using the system to describing the Long-tail entities in a real-world large dataset.
\end{enumerate}

The rest of this paper is organized in the following way. \todo{list the sections of the paper, and a paragraph before showing the results of the research}
